\documentclass[11pt,a4paper]{article}

% -------- Page & spacing (3-page safe) --------
\usepackage[left=2.5cm,right=2.5cm,top=1.8cm,bottom=1.8cm]{geometry}
\usepackage{setspace}
\setstretch{1.05}

\usepackage{titlesec}
\titlespacing*{\section}{0pt}{1.2ex}{0.8ex}
\titlespacing*{\subsection}{0pt}{1.0ex}{0.6ex}

\usepackage{amsmath}
\raggedbottom
% --------------------------------------------

\begin{document}
	
	\begin{center}
		\Large \textbf{Project Group 8: CIFAR-10 Object Recognition} \\[0.2cm]
		\normalsize
		Team Name: Atomax \\[0.1cm]
		Members: Kunal Verma, Debojit Roy Chowdhury, Moulindu Mapa
	\end{center}
	
	\vspace{0.3cm}
	
	\section{Problem Statement}
	
	The objective of this project is to develop a Convolutional Neural Network (CNN) for object
	recognition using the CIFAR-10 dataset. The task involves classifying input images into one
	of ten predefined object categories based on visual features learned during training.
	
	The project aims to demonstrate an understanding of deep learning concepts such as
	convolution, pooling, feature extraction, and classification, rather than achieving
	state-of-the-art performance. In addition to evaluating the model on the standard test
	dataset, the trained network is further validated using images obtained from the internet
	to assess its generalization capability on unseen real-world data.
	
	\section{Dataset Description}
	
	The dataset used in this project is the CIFAR-10 dataset, a widely used benchmark dataset
	for image classification tasks. The dataset consists of 60,000 color images of size
	$32 \times 32$ pixels, distributed equally across ten object classes: airplane, automobile,
	bird, cat, deer, dog, frog, horse, ship, and truck.
	
	The dataset is divided into 50,000 training images and 10,000 testing images. Each image
	is represented as an RGB image with three color channels. Prior to training, pixel values
	are normalized to the range $[0,1]$ to improve numerical stability during optimization.
	Class labels are converted into one-hot encoded vectors to enable multi-class
	classification using categorical cross-entropy loss.
	
	The CIFAR-10 dataset is chosen due to its balanced class distribution and suitability for
	evaluating convolutional neural networks.
	
	\section{Methodology}
	
	The methodology adopted in this project is based on supervised deep learning using a
	Convolutional Neural Network (CNN), which is well suited for image classification tasks
	due to its ability to automatically learn spatial hierarchies of features from raw pixel
	data.
	
	The proposed approach begins with preprocessing the input images through normalization.
	The normalized images are passed through convolutional layers that extract low-level and
	progressively higher-level visual features, which are later used for classification.
	
	Max-pooling layers are employed after convolutional layers to reduce the spatial
	dimensions of feature maps, thereby decreasing computational complexity and providing a
	degree of translation invariance. The extracted features are subsequently flattened and
	fed into fully connected layers that perform the final classification task.
	
	The network is trained using the backpropagation algorithm with categorical
	cross-entropy as the loss function and optimized using the Adam optimizer. Model
	performance is evaluated on a held-out test dataset as well as on external images
	collected from the internet.
	
	\section{Experimental Details}
	
	\subsection{Architecture Description}
	
	The convolutional neural network designed for this project follows a sequential
	architecture consisting of convolutional, pooling, and fully connected layers. The input
	to the network is a $32 \times 32$ RGB image.
	
	The first convolutional layer applies multiple filters with a small receptive field to
	capture low-level visual features such as edges and textures. This is followed by a
	max-pooling layer that reduces the spatial dimensions of the feature maps while retaining
	the most significant information. A second convolutional layer is then used to extract
	higher-level features, followed again by max-pooling to further reduce dimensionality.
	
	The resulting feature maps are flattened and passed through a fully connected layer with
	ReLU activation, enabling the network to learn non-linear combinations of the extracted
	features. The final output layer consists of ten neurons with a softmax activation
	function, corresponding to the ten classes in the CIFAR-10 dataset.
	
	A detailed summary of the network architecture is provided using TensorFlow’s
	\texttt{model.summary()} function.
	
	\subsection{Evaluation Metrics}
	
	The performance of the proposed model is evaluated using categorical cross-entropy as
	the loss function. Classification accuracy is used as the primary evaluation metric.
	
	\subsection{Dataset Split}
	
	The CIFAR-10 dataset is divided into a training set of 50,000 images and a test set of
	10,000 images. The training set is used for learning model parameters, while the test set
	is used to evaluate generalization performance.
	
	\subsection{Implementation Details}
	
	The model is implemented using the TensorFlow and Keras deep learning framework.
	Training is performed using the Adam optimizer. The network is trained for a limited
	number of epochs with a fixed batch size.
	
	All experiments are conducted on normalized image data, and model performance is
	evaluated on both the test dataset and external images.
	
	\subsection{Other Methods}
	
	No additional optimization techniques such as data augmentation, dropout, or batch
	normalization were employed. The model architecture was intentionally kept simple to
	maintain interpretability and originality.
	
	\section{Results and Discussion}
	
	\subsection{Results}
	
	The proposed convolutional neural network was trained for 15 epochs on the CIFAR-10
	training dataset. During training, the model showed a steady improvement in
	classification accuracy and a corresponding reduction in loss values.
	
	The final evaluation on the CIFAR-10 test dataset yielded a test accuracy of approximately
	69.24\%, with a final test loss of approximately 0.96. These results indicate reasonable
	generalization performance given the simplicity of the architecture.
	
	The trained model was also tested using external images obtained from the internet and
	was able to correctly predict object categories after appropriate preprocessing.
	
	\subsection{Visualizations}
	
	The training and validation accuracy and loss curves provide insight into the model’s
	learning behavior. Accuracy improved steadily across epochs, while loss values decreased
	gradually, indicating stable convergence without severe overfitting.
	
	\subsection{Performance Comparison}
	
	The convolutional neural network was compared with simpler baseline approaches typically
	used for image classification. The obtained test accuracy highlights the effectiveness of
	convolutional architectures for image recognition tasks.
	
	\subsection{Analysis}
	
	The results indicate that the model successfully captures discriminative visual features
	from the CIFAR-10 dataset. Some misclassifications were observed among visually similar
	classes due to low image resolution and limited network depth. Nevertheless, the model
	demonstrates robust performance given its simple design.
	
	\subsection{Improvements}
	
	Future work may explore deeper architectures, regularization techniques, data
	augmentation, and transfer learning to further improve performance.
	
	\section{References}
	
	\begin{itemize}
		\item Krizhevsky, A., Hinton, G. (2009). \textit{Learning Multiple Layers of Features from
			Tiny Images}. CIFAR-10 Dataset.\\
		\texttt{https://www.cs.toronto.edu/~kriz/cifar.html}
		
		\item Kaggle. \textit{CIFAR-10 Object Recognition Competition}.\\
		\texttt{https://www.kaggle.com/competitions/cifar-10}
	\end{itemize}
	
\end{document}
